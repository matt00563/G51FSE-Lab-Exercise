\section{Adam White}
My strengths: I was motivated to get a high quality of work and help my team achieve their best. Yes, it does sound cliche but we all want good grades. I was 
also very willing to learn and did not assume expertise in any of the subject areas.
My weaknesses: I felt that sometimes I would only communicate with one individual of the team at a time, rather than the whole group. Towards the end of the last project my attention span faltered in the face of JUnit testing.

Please do not forget:
\begin{itemize}
	\item First paragraph should have your comments about the module
	\item Second one, a selfie img with Max
	\item Last one, what you learned in this module.
\end{itemize}

\subsection{Comments about the module}
\begin{figure}
	\begin{tabular}[|l | l|]
	\textbf{Pros} & \textbf{Cons} \\ \hline
	Well taught by Max & last group exercise far too time consuming \\ \hline
	Module was well organised & \\ \hline 
	Skills labs were helpful & \\ \hline
	\end{tabular}
\end{figure}

\subsection{Selfie with Max}

\begin{figure}[h]
\caption{Selfie with Max}
\centering
\includegraphics[width=0.5\textwidth]{}
\label{fig:selfie}
\end{figure}

You can then use the label of the figure to reference it later with the command ${\backslash}ref$. you can comment out the next line to see an example of how it works.

My selfie with Max is in  Figure~\ref{fig:selfie}.

\subsection{What I have learned in this module}
I have learnt more about different process models, loads about requirements and their importance, agile methods and vigorous testing. I have learnt more UML's than I have fingers. But most importantly:\textbf{I actually understand what software engineering is!} The process of making good software, and not just 'software design', as I'd previously thought. 